\documentclass[11pt,]{article}
\usepackage{lmodern}
\usepackage{amssymb,amsmath}
\usepackage{ifxetex,ifluatex}
\usepackage{fixltx2e} % provides \textsubscript
\ifnum 0\ifxetex 1\fi\ifluatex 1\fi=0 % if pdftex
  \usepackage[T1]{fontenc}
  \usepackage[utf8]{inputenc}
\else % if luatex or xelatex
  \ifxetex
    \usepackage{mathspec}
    \usepackage{xltxtra,xunicode}
  \else
    \usepackage{fontspec}
  \fi
  \defaultfontfeatures{Mapping=tex-text,Scale=MatchLowercase}
  \newcommand{\euro}{€}
\fi
% use upquote if available, for straight quotes in verbatim environments
\IfFileExists{upquote.sty}{\usepackage{upquote}}{}
% use microtype if available
\IfFileExists{microtype.sty}{%
\usepackage{microtype}
\UseMicrotypeSet[protrusion]{basicmath} % disable protrusion for tt fonts
}{}
\usepackage[margin=1in]{geometry}
\ifxetex
  \usepackage[setpagesize=false, % page size defined by xetex
              unicode=false, % unicode breaks when used with xetex
              xetex]{hyperref}
\else
  \usepackage[unicode=true]{hyperref}
\fi
\hypersetup{breaklinks=true,
            bookmarks=true,
            pdfauthor={},
            pdftitle={rapr: Representative and Adequate Prioritisations in R},
            colorlinks=true,
            citecolor=blue,
            urlcolor=blue,
            linkcolor=magenta,
            pdfborder={0 0 0}}
\urlstyle{same}  % don't use monospace font for urls
\setlength{\parindent}{0pt}
\setlength{\parskip}{6pt plus 2pt minus 1pt}
\setlength{\emergencystretch}{3em}  % prevent overfull lines
\setcounter{secnumdepth}{0}

%%% Use protect on footnotes to avoid problems with footnotes in titles
\let\rmarkdownfootnote\footnote%
\def\footnote{\protect\rmarkdownfootnote}

%%% Change title format to be more compact
\usepackage{titling}

% Create subtitle command for use in maketitle
\newcommand{\subtitle}[1]{
  \posttitle{
    \begin{center}\large#1\end{center}
    }
}

\setlength{\droptitle}{-2em}
  \title{rapr: Representative and Adequate Prioritisations in R}
  \pretitle{\vspace{\droptitle}\centering\huge}
  \posttitle{\par}
  \author{Jeffrey O. Hanson$^1$, Jonathan R. Rhodes$^2$, Hugh P. Possingham$^1$,
Richard A. Fuller$^1$\\$^1$School of Biological Sciences, The University
of Queensland, Brisbane, QLD, Australia\\$^2$School of Geography,
Planning and Environmental Management, The University of Queensland,
Brisbane, QLD, Australia\\Correspondance should be addressed to
\href{mailto:jeffrey.hanson@uqconnect.edu.au}{jeffrey.hanson@uqconnect.edu.au}}
  \preauthor{\centering\large\emph}
  \postauthor{\par}
  \predate{\centering\large\emph}
  \postdate{\par}
  \date{11 January 2016}



\begin{document}

\maketitle

\begin{abstract}
A central aim in conservation is to maximise the long-term persistence
of biodiversity. To fulfil this aim, reserve networks are used to
safeguard biodiversity patterns (eg. species, populations) and processes
(eg. evolutionary processes that underpin genetic variation). Reserve
selection is often formulated as an optimisation problem to identify
cost-effective prioritisations. However, most existing decision support
tools are based on formulations that are well suited for preserving
biodiversity patterns, but not biodiversity processes. To fill this gap
in the conservation planning toolbox, we developed the \texttt{rapr} R
package. This R package provides functions to solve reserve selection
problems using two novel formulations. Here, we explore the
functionality of this R package using a conservation planning exercise
in Queensland, Australia as a case-study. We found that explicitly
considering biodiversity processes can alter a prioritisation. However,
we also found that only a few additional planning units are required to
sufficiently preserve biodiversity processes. Our research highlights
the need to explicitly consider biodiversity patterns and processes
simultaneously in conservation planning.
\end{abstract}



\end{document}
